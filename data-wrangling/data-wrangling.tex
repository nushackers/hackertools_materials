\documentclass[12pt]{beamer}

\usetheme[sectionpage=none, subsectionpage=progressbar, progressbar=foot, numbering=fraction]{metropolis}

\usepackage{tabularx}

\makeatletter
\setlength{\metropolis@frametitle@padding}{1.6ex}% <- default 2.2 ex

\setbeamertemplate{footline}{%
  \begin{beamercolorbox}[wd=\textwidth, sep=1.5ex]{footline}% <- default 3ex
    \usebeamerfont{page number in head/foot}%
    \usebeamertemplate*{frame footer}
    \hfill%
    \usebeamertemplate*{frame numbering}
  \end{beamercolorbox}%
}
\makeatother

\AtBeginSubsection
{
  \begin{frame}{Where are we?}
    \tableofcontents[sectionstyle=show/shaded, subsectionstyle=show/shaded/hide]
  \end{frame}
}

\makeatletter
\setbeamertemplate{headline}{
  \begin{beamercolorbox}{upper separation line head}
  \end{beamercolorbox}
  \begin{beamercolorbox}{section in head/foot}
    \vskip2pt\insertsectionnavigationhorizontal{\paperwidth}{}{}\vskip2pt
  \end{beamercolorbox}
  \begin{beamercolorbox}{lower separation line head}
  \end{beamercolorbox}
}
\makeatother
\setbeamercolor{section in head/foot}{fg=normal text.bg, bg=structure.fg}

\setbeamertemplate{itemize items}[square]


\usepackage{menukeys}
\usepackage{minted}
\setminted[bash]{fontsize=\small, tabsize=2, breaklines}

\title{Hacker Tools: \\Data Wrangling \\ (Resources developed by Julius)}
\author{Noel Kwan}
\date{06 September 2022 \\ Resources at \url{https://bit.ly/3BeEJvg}}

\begin{document}

\frame[plain]{\titlepage}

\section{Introduction}
\subsection{}

\begin{frame}{NUS Hackers}

  \begin{center}
    \includegraphics[width=0.5\linewidth]{../NUSHackers}

    \url{http://nushackers.org}
  \end{center}

  \begin{center}
    \textbf{hacker}school

    Friday \textbf{Hacks}

    \textbf{Hack} \& Roll

    \textbf{Hacker} Tools
  \end{center}

\end{frame}

\begin{frame}{About Me}
  Hi! I'm Noel. My GitHub is \url{https://github.com/kwannoel}

  A Year 4 Computer Science Undergraduate who loves hacking and building systems.

  I enjoy retro gaming, board games and learning about programming languages.

\end{frame}

\begin{frame}{Required Software}
  Unix-like environment, either one of these:
  \begin{itemize}
    \item Linux (you're good if you attended and installed Linux during our Linux Install Fest)
    \item macOS\footnote{Open Terminal, and run \mintinline{bash}{xcode-select --install} first}
    \item BSD
    \item Other Unix-like OS'es (Minix, Solaris, AIX, HP-UX, etc.)
    \item WSL (Windows Subsystem for Linux) should also be alright, but no guarantee
  \end{itemize}
\end{frame}

\begin{frame}{What is Data Wrangling?}
  \begin{itemize}
    \item Have you ever had a bunch of text and wanted to do something with it?
    \item Great! That's \textbf{Data Wrangling}
    \item Adapting data from one format to another, until you end up with exactly what you wanted.
  \end{itemize}
\end{frame}


\begin{frame}{The Unix Philosophy}
  \begin{enumerate}
    \item Write programs that do one thing and do it well.
    \item Write programs to work together.
    \item Write programs to handle text streams, because that is a universal interface.
  \end{enumerate}
\end{frame}

\begin{frame}[fragile]{Basic Data Wrangling (1/2)}
  Linux:
  \begin{minted}{bash}
cat log | grep -i "Mar 21 13:01"
  \end{minted}
  \begin{itemize}
    \item This is an example of basic data wrangling: finding all system log entries that mentions Intel
    \item Most of data wrangling is just about knowing what tools you have, and how to combine them.
    \item Remember \textbf{The Unix Philosophy}!
  \end{itemize}
\end{frame}

\begin{frame}{Basic Data Wrangling (2/2)}
  \begin{itemize}
    \item Let's start from the beginning:
          \begin{enumerate}
            \item We need a data source
            \item Something to do with it.
          \end{enumerate}
    \item A good use case is for logs, because you often want to investigate them, but reading the whole thing is not feasible.
  \end{itemize}
\end{frame}

\begin{frame}[fragile]{Data Wrangling Example (1/2)}
  Let's try to figure out who is trying to log into my server.
  \begin{itemize}
    \item First, I try to look into my server's log: \\\mintinline{bash}{cat log}
    \item That's far too much stuffs!
    \item Let's limit it to \texttt{ssh} stuffs: \\\mintinline{bash}{cat log | grep sshd}
    \item That is still way more stuffs than what we wanted, and it's pretty hard to read.
  \end{itemize}
\end{frame}

\begin{frame}[fragile]{Data Wrangling Example (2/2)}
  We can do better!
  \begin{minted}{bash}
cat log
| grep sshd
| grep "Accepted publickey for"
  \end{minted}
  There's still a lot of noise here.

  There are \emph{a lot} of ways to get rid of that, but let’s look at one of the most powerful tools in your toolkit: \texttt{sed}.
\end{frame}

\begin{frame}{Exploring Codebase}
\end{frame}

\section{sed and Regular Expression (regex)}
\subsection{}
\begin{frame}{sed? Isn't that the adjective to describe my life?}
  \begin{itemize}
    \item \texttt{sed} is a stream editor that builds on top of the old \texttt{ed}\footnote{If you're into lame computing jokes, here's a joke about \texttt{ed}: \url{https://www.gnu.org/fun/jokes/ed-msg.html}} editor
    \item In it, you basically give short commands for how to modify the file.
    \item If you use \texttt{vim}, you should be familiar with some of the commands (\texttt{ed} -> \texttt{vi} -> \texttt{vim})
    \item There are tonnes of commands, but the most common one is \texttt{s} for substitution.
  \end{itemize}
\end{frame}

\begin{frame}[fragile]{Back to Our Example}
  \begin{minted}{bash}
cat log
| grep sshd
| grep "Accepted publickey for"
| sed 's/.*Accepted publickey for //'
  \end{minted}
  \begin{itemize}
    \item Wow! It's a lot cleaner.
    \item What we just wrote was a simple \textbf{Regular Expression}
  \end{itemize}
\end{frame}

\begin{frame}{The s Command in sed}
  Syntax: \texttt{s/REGEX/SUBSTITUTION/}
  \begin{itemize}
    \item \texttt{REGEX} is the regular expression you want to search for.
    \item \texttt{SUBSTITUTION} is the text you want to substitute matching text with.
  \end{itemize}
\end{frame}

\begin{frame}{What is Regular Expression}
  \begin{itemize}
    \item It's a powerful construct that lets you match text against patterns.
    \item They are common and useful enough that it’s worthwhile to take some time to understand how they work.
    \item Usually (though not always) surrounded by \texttt{/}
    \item Most characters just carry their normal meaning, but some characters have special matching behaviour.
    \item Exactly which characters do what vary somewhat between different implementations of regular expressions, which is a source of great frustration.
  \end{itemize}
\end{frame}

\begin{frame}{List of Regex Special Characters}
  \begin{tabular}{|c|c|}
    \hline
    \textbf{Character} & \textbf{Meaning}                                            \\ \hline
    \texttt{.}         & Any single character except newline                         \\ \hline
    \texttt{*}         & Zero or more of the preceding match                         \\ \hline
    \texttt{?}         & One or more of the preceding match                          \\ \hline
    \texttt{[abc]}     & Any one character of \texttt{a}, \texttt{b}, and \texttt{c} \\ \hline
    \texttt{(RX1|RX2)} & Either something that matches \texttt{RX1} or \texttt{RX2}  \\ \hline
    \texttt{\^{}}      & The start of the line                                       \\ \hline
    \texttt{\$}        & The end of the line                                         \\ \hline
  \end{tabular}

  If you are unfamiliar with regex, there is a nice tutorial at \url{https://regexone.com/}
\end{frame}

\begin{frame}{Obsolete vs Modern Regex}
  \begin{itemize}
    \item Note that \texttt{sed}'s regex is somewhat weird and will require you to put a \texttt{\textbackslash} before most of these to give them special meaning.
    \item This is because by default \texttt{sed} is using the \emph{obsolete} regex format.
    \item You can avoid this problem by passing \texttt{-E} flag to \texttt{sed}, which tells it to switch to the \emph{modern} regex format.
    \item You can explore the differences by running \\\mintinline{bash}{man re_format}
  \end{itemize}
\end{frame}

\begin{frame}{Looking at our regex just now}
  \texttt{/.*Accepted publickey for /}

  \begin{itemize}
    \item It means any text that starts with any number of characters, followed by the literal string \texttt{"Accepted publickey for "}
    \item However, regexes are tricky.
    \item What if the username is also \texttt{"Accepted publickey for "}?
    \item Why? By default, \texttt{*} and \texttt{+} are ``greedy'' -- they will match as much text as they can
  \end{itemize}
\end{frame}

\begin{frame}[fragile]{Solution: Match the whole line}
  \begin{minted}{bash}
| sed -E 's/.*Accepted publickey for (.*) from ([0-9]{1,3}\.[0-9]{1,3}\.[0-9]{1,3}\.[0-9]{1,3}) port ([0-9]+) ssh2: RSA SHA256:.*//'
  \end{minted}
  Let's look at what's going on with a regex debugger\footnote{\url{https://regex101.com/r/wPc8Ii/3}}
\end{frame}

\begin{frame}{Explanation}
  \begin{itemize}
    \item The start is still as before.
    \item Then on any string of characters (username).
    \item Then on \texttt{from} followed by an IP address\footnote{This matches \texttt{999.999.999.999} which is not a valid IPv4 address. A regex that only matches valid address is left as an exercise}
    \item Then on \texttt{port} followed by a sequence of digits.
    \item Finally, we try to match on the suffix \texttt{ssh2: RSA SHA256:} followed by any string of characters.
    \item Notice that with this technique, a username of \texttt{Accepted publickey for} will not confuse us anymore. Can you see why?
  \end{itemize}
\end{frame}

\begin{frame}{Capture Groups}
  \begin{itemize}
    \item Oh no, the entire log is now empty.
    \item We want to keep the username
    \item Use \textbf{Capture Groups}!
    \item Any text matched by a regex surrounded by parentheses is stored in a numbered capture group.
    \item Capture group 0 is special. It is the whole text matched by the regex.
    \item These are available in the \texttt{SUBSTITUTION}\footnote{In some engines, even in the pattern itself!} as \texttt{\textbackslash 1}, \texttt{\textbackslash 2}, \texttt{\textbackslash 3}, etc.
  \end{itemize}
\end{frame}

\begin{frame}[fragile]{Using Capture Groups in sed}
  \begin{minted}{bash}
| sed -E 's/.*Accepted publickey for (.*) from ([0-9]{1,3}\.[0-9]{1,3}\.[0-9]{1,3}\.[0-9]{1,3}) port ([0-9]+) ssh2: RSA SHA256:.*/\1/'
  \end{minted}
  \begin{itemize}
    \item Note that in our current regex, capture group 1 is username, capture group 2 is IP address, capture group 3 is port number.
    \item You can try out using \texttt{\textbackslash 2} and \texttt{\textbackslash 3} instead of \texttt{\textbackslash 1}.
  \end{itemize}
\end{frame}

\begin{frame}{More on Regular Expressions}
  \begin{itemize}
    \item As you can probably imagine, you can come up with \emph{really} complicated regex.
    \item For example, there is an article on how you might match an email address\footnote{\url{https://www.regular-expressions.info/email.html}}. It's not easy\footnote{\url{http://emailregex.com/}}. People have even written tests\footnote{\url{https://fightingforalostcause.net/content/misc/2006/compare-email-regex.php}} and test matrices\footnote{\url{https://mathiasbynens.be/demo/url-regex}}
    \item Regular expressions are notoriously hard to get right, but they are also very handy to have in your toolbox!
  \end{itemize}
\end{frame}

\begin{frame}{More Regex Trivia}
  \begin{itemize}
    \item You can check for prime numbers using regex\footnote{\url{https://www.noulakaz.net/2007/03/18/a-regular-expression-to-check-for-prime-numbers/}}
    \item You can match \texttt{A B C} where $A + B = C$\footnote{\url{http://www.drregex.com/2018/11/how-to-match-b-c-where-abc-beast-reborn.html}}
    \item You can match nested brackets, e.g. to parse Lisp's s-expressions using Regex\footnote{\url{http://www.drregex.com/2017/11/match-nested-brackets-with-regex-new.html}}
    \item Note: these are more for curiosity purposes. There are usually better tools than regex, although for a quick and dirty script, regex is usually enough.
  \end{itemize}
\end{frame}

\begin{frame}[fragile]{Back to Data Wrangling}
  So now we have
  \begin{minted}{bash}
cat log
| grep sshd
| grep "Accepted publickey for"
| sed -E 's/.*Accepted publickey for (.*) from ([0-9]{1,3}\.[0-9]{1,3}\.[0-9]{1,3}\.[0-9]{1,3}) port ([0-9]+) ssh2: RSA SHA256:.*/\1/'
  \end{minted}
\end{frame}

\begin{frame}[fragile]{sed All the Way!}
  But we can do everything just with sed!
  \begin{minted}{bash}
cat log
| sed -E -e '/Accepted publickey for/!d' -e 's/.*Accepted publickey for (.*) from ([0-9]{1,3}\.[0-9]{1,3}\.[0-9]{1,3}\.[0-9]{1,3}) port ([0-9]+) ssh2: RSA SHA256:.*/\1/'
  \end{minted}
  \begin{itemize}
    \item \texttt{d} is to delete, \texttt{!} is to apply the function to the lines \textbf{not} selected by the pattern.
    \item Check out \mintinline{bash}{man sed}!
  \end{itemize}
\end{frame}

\section{More Advanced Data Wrangling}
\subsection{}
\begin{frame}[fragile]{Let's look for common usernames}
  \begin{minted}{bash}
| sort | uniq -c
  \end{minted}
  \begin{itemize}
    \item \texttt{sort} will, well, sort its input.
    \item \texttt{uniq -c} will collapse consecutive lines that are the same into a single line, prefixed with a count of the number of occurrences.
  \end{itemize}
\end{frame}

\begin{frame}[fragile]{How about the most common logins?}
  \begin{minted}{bash}
| sort -nk1,1 | tail -n3
  \end{minted}
  \begin{itemize}
    \item \texttt{-n} sorts in numeric (instead of lexicographic) order
    \item \texttt{-k1} means sort only by the first whitespace separated column
    \item \texttt{,n} means sort until the \texttt{n}th field, where the default is the end of the line\footnote{In this example, sorting by the whole line wouldn’t matter}.
    \item \textbf{Exercise}: what if we wanted the least common ones?\pause
    \item Either use \texttt{head} instead of \texttt{tail} or use \texttt{sort -r} which sorts in reverse order.
  \end{itemize}
\end{frame}

\begin{frame}[fragile]{We can do better}
  Okay, so that’s pretty cool, but we’d sort of like to only give the usernames, and maybe not one per line?
  \begin{minted}{bash}
| awk '{print $2}' | paste -sd, -
  \end{minted}
  Let's start with \texttt{paste}
  \begin{itemize}
    \item It lets you combine lines (\texttt{-s}) by a given single-character delimiter (\texttt{-d}), and ask it to to read from \texttt{STDIN} (\texttt{-})\footnote{Using GNU paste, the \texttt{-} can be omitted, but this is not POSIX compliant.}
    \item You can also emulate this using \mintinline{bash}{tr '\n' ','}, but this results in a trailing comma.
  \end{itemize}
\end{frame}

\begin{frame}{awk}
  \begin{itemize}
    \item A programming language that happens to be really good at processing text streams.
    \item There is \emph{a lot} to say about \texttt{awk} if you were to learn it properly, but as with many other things here, we’ll just go through the basics.
  \end{itemize}
\end{frame}

\begin{frame}{awk Syntax}
  \begin{itemize}
    \item Basic \texttt{awk} syntax: \texttt{pattern \{ block \}}
    \item \texttt{awk} takes in an optional pattern plus a block saying what to do if the pattern matches a given line.
    \item The default pattern (if no pattern is provided) matches all lines.
    \item Inside the block, \texttt{\$0} is set to the entire line's content, and \texttt{\$1} to \texttt{\$n} is set to the n-th field of that line, when separated by \texttt{awk} field separator\footnote{whitespace by default, can be changed with \texttt{-F}}.
  \end{itemize}
\end{frame}

\begin{frame}[fragile]{Our Use of awk}
  \begin{minted}{bash}
| awk '{print $2}'
  \end{minted}
  \begin{itemize}
    \item So in this case, we're saying that, for every line, print the contents of the second field, which happens to be the username.
  \end{itemize}
\end{frame}

\begin{frame}[fragile]{More fancy awk}
  Let’s compute the number of single-use usernames that start with \texttt{r} and end with \texttt{t}:
  \begin{minted}{bash}
| awk '$1 == 1 && $2 ~ /^r[^ ]*t$/ { print $2 }' | wc -l
  \end{minted}

  Let's unpack this!
  \begin{itemize}
    \item The pattern means the first field of the line should be equal to \texttt{1} (the count from \texttt{uniq -c}), and the second field should match the regex.
    \item The block says to print the second field (username)
    \item Finally, we count the number of lines in the output with \texttt{wc -l}.
  \end{itemize}
\end{frame}

\begin{frame}[fragile]{awk as a Programming Language}
  Remember that \texttt{awk} is a programming language, so we can actually not use \texttt{wc -l} at all:
  \begin{minted}{awk}
BEGIN { rows = 0 }
$1 == 1 && $2 ~ /^r[^ ]*t$/ { rows += $1 }
END { print rows }
  \end{minted}
  \begin{itemize}
    \item \texttt{BEGIN} is a pattern that matches the start of the input, and \texttt{END} matches the end.
    \item First we initialise the count to 0. The per-line block just adds the count from the first field. Then we print it out at the end.
  \end{itemize}
\end{frame}

\begin{frame}{Advanced awk}
  \begin{itemize}
    \item In fact, we could get rid of \texttt{grep} and \texttt{sed} entirely, because \texttt{awk} can do it all, but that is left as an exercise.
    \item A good resource to read is \url{https://backreference.org/2010/02/10/idiomatic-awk/}
  \end{itemize}
\end{frame}

\begin{frame}[fragile]{We can do Maths too!}
  \begin{minted}{bash}
| awk '{print $1}'
| paste -sd+ -
| bc
  \end{minted}
  \begin{itemize}
    \item \texttt{bc} is actually a calculator language.
    \item You can even run it straight from your shell and use it as a normal calculator.
    \item In this case, we are piping a mathematical expression to \texttt{bc}
  \end{itemize}
\end{frame}

\begin{frame}{Data Wrangling to Make Arguments (1/2)}
  \begin{itemize}
    \item \textbf{Exercise}: find out what the \texttt{xargs} tool does (hint: try to pipe to \texttt{xargs echo})\pause
    \item Since we can pipe data to it, we can use data wrangling to make arguments too.
    \item Say we want to delete all files that matches the regex \texttt{asd.a [0-9]\{2\}}
  \end{itemize}
  \mintinline{bash}{ls | grep -E 'asd.a [0-9]{2}' | xargs rm}

  What happened?
\end{frame}

\begin{frame}[fragile]{Data Wrangling to Make Arguments (2/2)}
  \begin{itemize}
    \item It's the annoying whitespace splitting again.
    \item A workaround is to use the null character (\texttt{\textbackslash 0}) as delimiter instead
  \end{itemize}
  \begin{minted}{bash}
ls
| grep -E 'asd.a [0-9]{2}'
| tr '\n' '\0'
| xargs -0 rm
  \end{minted}
\end{frame}

\section{Exercises}
\begin{frame}{Exercises (1/2)}
  \begin{itemize}
    \item How is \mintinline{bash}{sed s/REGEX/SUBSTITUTION/g} different from regular \texttt{sed}? What about \texttt{/i}?
    \item To do in-place substitution it is quite tempting to do something like \mintinline{bash}{sed s/REGEX/SUBSTITUTION/ input.txt > input.txt}. However this is a bad idea, why? Is this particular to \texttt{sed}?
  \end{itemize}
\end{frame}

\begin{frame}{Exercises (2/2)}
  \begin{itemize}
    \item Find the number of words (in \mintinline{bash}{/usr/share/dict/words}) that contain at least three \texttt{a}s and don’t have \texttt{ness} ending.
    \item What are the three most common last two letters of those words?
    \item How many unique two-letter combinations are there?
    \item And for a challenge: which combinations do not occur?
          % cat /usr/share/dict/words | grep -v "ness" | grep '.*a.*a.*a.*' | grep -E -o '.{2}$' | sort | uniq -c | sort -nk1,1 | tail -n3
          % cat /usr/share/dict/words | grep -v "ness" | grep '.*a.*a.*a.*' | grep -E -o '.{2}$' | sort | uniq | wc -l
          % cat /usr/share/dict/words | grep -v "ness" | grep '.*a.*a.*a.*' | grep -E -o '.{2}$' | sort | uniq | comm -1 -3 - <(echo {a..z}{a..z} | tr ' ' '\n') | paste -sd, -
  \end{itemize}
\end{frame}


\section{Conclusion}
\subsection{}
\begin{frame}
  \frametitle{Talk to us!}
  \begin{itemize}
    \item \textbf{Feedback form}: \url{https://bit.ly/ht3-feedback}
    \item \textbf{Upcoming Hacker Tools}:

          Tue, 5th Oct 2020, 7.00pm -- LaTeX
  \end{itemize}
\end{frame}

\end{document}
