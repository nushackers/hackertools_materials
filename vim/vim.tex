\documentclass[12pt]{beamer}

\usetheme[sectionpage=none, subsectionpage=progressbar, progressbar=foot, numbering=fraction]{metropolis}

\usepackage{tabularx}
\usepackage{verbatim}

\makeatletter
\setlength{\metropolis@frametitle@padding}{1.6ex}% <- default 2.2 ex

\setbeamertemplate{footline}{%
  \begin{beamercolorbox}[wd=\textwidth, sep=1.5ex]{footline}% <- default 3ex
    \usebeamerfont{page number in head/foot}%
    \usebeamertemplate*{frame footer}
    \hfill%
    \usebeamertemplate*{frame numbering}
  \end{beamercolorbox}%
}
\makeatother

\AtBeginSubsection
{
  \begin{frame}{Where are we?}
    \tableofcontents[sectionstyle=show/shaded, subsectionstyle=show/shaded/hide]
  \end{frame}
}

\makeatletter
\setbeamertemplate{headline}{
  \begin{beamercolorbox}{upper separation line head}
  \end{beamercolorbox}
  \begin{beamercolorbox}{section in head/foot}
    \vskip2pt\insertsectionnavigationhorizontal{\paperwidth}{}{}\vskip2pt
  \end{beamercolorbox}
  \begin{beamercolorbox}{lower separation line head}
  \end{beamercolorbox}
}
\makeatother
\setbeamercolor{section in head/foot}{fg=normal text.bg, bg=structure.fg}

\setbeamertemplate{itemize items}[square]


\usepackage{menukeys}
\usepackage{minted}
\setminted[bash]{fontsize=\small, tabsize=2, breaklines}

\title{Hacker Tools: \\Vim}
\author{Lam Chun Yu}
\date{18 October 2022 \\ Slides at \url{https://hckr.cc/ht-vim-slides}}
% TODO: Add link to slides

\begin{document}

\frame[plain]{\titlepage}

\section{Introduction}
\subsection{}

\begin{frame}{NUS Hackers}

  \begin{center}
    \includegraphics[width=0.5\linewidth]{../NUSHackers}

    \url{http://nushackers.org}
  \end{center}

  \begin{center}
    \textbf{hacker}school

    Friday \textbf{Hacks}

    \textbf{Hack} \& Roll

    \textbf{Hacker} Tools
  \end{center}

\end{frame}

\begin{frame}{About Me}
  Hi! I'm Chun Yu. My GitHub is \url{https://github.com/gunbux}

  A Year 2 Computer Science Undergraduate who loves learning about operating systems and building cool stuff.

  I enjoy playing the violin, and first person shooters.

\end{frame}

\begin{frame}{Required Software}
  You need \textbf{vim} (that's what we're learning today)
\end{frame}

\begin{frame}{Why do we want to \underline{learn} an editor?}
  \begin{itemize}
    \item We don't have workshops on how to learn a web browser, so why are editors important?
    \item Writing code, or editing files on a computer has a lot of moving parts: you spend a lot more time switching files, reading, navigating, editing code compared to writing a long stream of words sometimes.
    \item As a power user, you probably will spend a lot of time doing these things, it might be good to find an editor that will \textbf{speed up your workflow.}
  \end{itemize}
\end{frame}

\begin{frame}{How to learn an editor}
    \begin{itemize}
        \item Start with the \textbf{fundamentals} (this is what we'll cover)
        \item \textbf{Practice} as much as possible
        \item Within about 10-20 hours of use, you'll be back to your normal speed
        \item After that your new editor should start to save you lots of time
    \end{itemize}{}
\end{frame}{}

\begin{frame}{What to expect today?}
        Today we'll be covering one such editor, \textbf{vim}.

        The idea is to kickstart you knowledge, and give you the \underline{fundamentals and resources to go off on your own.}
\end{frame}{}

\begin{frame}{Which editor to learn?}
    \begin{itemize}
        \item People tend to have very strong opinions on this. See \url{https://en.wikipedia.org/wiki/Editor_war}
        \item Vi(m) vs Emacs
        \item It’s completely up to you! we’ll cover one of the popular options out there, vim but you can also look up Emacs and choose one that interests you most

        \begin{center}
            \includegraphics[width=0.5\linewidth]{2022hs5/editor-curve.png} 
        \end{center}{}
    \end{itemize}{}
\end{frame}{}

\begin{frame}{Origins of Vi(m) (1/2)}
    \begin{itemize}
        \item Vim stands for vi iMitation, later changed to vi iMproved
        \item Created by Bram Moolenaar in 1991.
        \item It is based off another text editor, \textbf{vi}, created by Bill Joy in 1976.
    \end{itemize}{}
\end{frame}{}

\begin{frame}{Origins of Vi(m) (2/2)}
    \begin{itemize}
        \item Bill Joy was trying to create an editor that was usable with a 300 baud modem. (approximately 0.3 kbit/second today). Essentially, you could only type one letter a second, so the commands had to be really short.
        \item As it turns out, when you optimize heavily to minimize the number of keystrokes, you have a
        \textbf{really efficient editor}
    \end{itemize}{}
\end{frame}{}

\begin{frame}{Benefits of Vim}
    \begin{itemize}
        \item Vi(m) is the de facto default editor in any Unix-based system. You can probably find vi/vim in 
        any and every Unix-based system today
        \item Vim is also extremely customizable and programmable
        \item There is a huge community and plugin support for almost anything imaginable
    \end{itemize}{}
\end{frame}{}

\begin{frame}{Philosophy of Vim}
    \begin{itemize}
        \item Vim is a \textbf{modal text editor}
        \item We have \underline{different modes} for inserting text vs manipulating text
        \item The idea here is that you tend to spend more time reading/making smaller edits, instead of
        writing big essays in one big go.
    \end{itemize}{}
\end{frame}{}

\section{Using Vim}
\subsection{}

\begin{frame}{Modes of Vim}
    There are a few primary modes of vim:
    \begin{itemize}
        \item \underline{Normal Mode} - For moving around a file and making small edits
        \item \underline{Insert Mode} - For inserting text
        \item \underline{Visual Mode} - For selecting blocks of text (plain, line or block)
        \item \underline{Command Mode} - For entering commands
    \end{itemize}{}
\end{frame}{}

\begin{frame}{Modes of Vim}
    Keystrokes have different meanings in different operating modes. for example, \keys{x} in insert mode will insert a literal character ‘x’, but in normal mode, it will delete the current selection
\end{frame}{}

\begin{frame}{Opening Vim and \underline{quitting it} (1/2)}
    There's a common joke that if you give a web designer a computer with vim loaded up, and ask them to quit vim, 
    you get a random string generator
    
    \includegraphics[width=1\linewidth]{2022hs5/exit_vim.png}
\end{frame}

\begin{frame}{Opening Vim and \underline{quitting it} (2/2)}
    To open vim, just type \texttt{vim} in the terminal. If you are using gVim, you can just open the executable.
    
    To close vim, type \texttt{:q}. We'll go through what this means later
\end{frame}{}

\begin{frame}{Changing modes}
    By default, you start vim in \textbf{normal mode}. For most cases, you will transition from normal mode to 
    another mode, based on your use case, and then return back to normal mode after.
\end{frame}{}

\begin{frame}{Normal to Command Mode}
    Command mode is where we run commands similar to a command line in vim.
    To go to this mode, simple press \keys{:}. A text bar should appear at the bottom of the screen. From there,
    we can execute several vim commands.
    \begin{itemize}
        \item \keys{:} - go to command mode
        \item \keys{q} (in command mode) - quits the file
        \item \keys{w} (in command mode) - saves the file
        \item \keys{!} - force an action
        \item \keys{:wq} - save the file then quit
        \item \keys{:q!} - force quit file without saving
    \end{itemize}{}
    Once you are done, vim should automatically put you back in normal mode
\end{frame}{}

\begin{frame}{Any mode to normal mode}
    Normal mode is your default mode where you should spend most of your time. In vim, if you ever get lost
    or are not sure what is happening, always \textbf{reset to normal mode}
    
    \textbf{
    \keys{Esc} will bring you to normal mode from any of the other modes. You will be using this a lot.
    }
\end{frame}{}

\begin{frame}{Why Escape?}
    It might seem quite counterintuitive to use escape since it's quite out of place on your keyboard. However, 
    vi was created using an \underline{ADM-3A terminal}. It looks like this:
    \begin{center}
        \includegraphics[width=0.75\linewidth]{2022hs5/adm-3a.png}
    \end{center}{}
    Some programmers also map \keys{Caps Lock} to \keys{Esc} or other mappings for convenience
\end{frame}

\begin{frame}{Changing modes}
    \begin{tabular}{|c|c|c|}
       \hline
       Mode  &  Description & Hotkey\\
       \hline
       Normal & Navigate the file & \keys{Esc}\\
       \hline 
       Insert & Inserting text & \keys{i}, \keys{I}, \keys{a}, \\ & & \keys{A}, \keys{o}, \keys{O}\\
       \hline
       Command & For entering commands & \keys{:}\\
       \hline
       Visual & For selecting text & \keys{v}, \keys{V}, \keys{Ctrl + v}\\
       \hline
    \end{tabular}{}
\end{frame}{}

\section{Normal Mode}
\subsection{}

\begin{frame}{Navigating in normal mode}
    First let's open up a file using vim by using \texttt{vim (filename)}.

    We can navigate the file by using \texttt{hjkl} (left, down, up, right respectively)

    Why not arrow keys (or a mouse)? Historically, it's because the old keyboards did not have arrow keys or a mouse. 
    
    However, in practice, it is extremely efficient as you don't need to move you hands away from the alphanumeric
    keys to do anything.
\end{frame}{}

\begin{frame}{Normal Mode}
    \begin{tabular}{|c|c|}
       \hline
       Type  &  Description\\
       \hline
       Basic &  \keys{hjkl}: left, down, up, right\\
       \hline 
       Word & \keys{w}: next word, \keys{b}: back a word\\
       \hline
       File & \keys{gg}: top of file, \keys{G}: bottom of file\\
       \hline
       Line & \keys{0}: beginning of line, \keys{\$}: end of line,\\ & \keys{\^{}} first non-whitespace of line\\
       \hline
       Line Numbers & \keys{34G}: to go to line 34 in the file\\
       \hline
       Screen & \keys{H}igh part of screen, \keys{M}iddle of screen, \\ & \keys{L}ow part of screen\\
       \hline
       Braces & \keys{\%} to go to corresponding braces\\
       \hline
       Repeating & \keys{10j}: to go down 10 times\\
       \hline
       Scroll & \keys{Ctrl + d}: to scroll down,\\ & \keys{Ctrl + u} to scroll up\\
       \hline
    \end{tabular}{}
\end{frame}{}

\begin{frame}{Searching}
    Searching is slightly different in vim, there are two main ways to search something:
    \begin{itemize}
        \item Find inline: \keys{f} to find further up in the line, \keys{F} to find in everything before the cursor
        \item Search in file: \keys{/ + query} to search forward in the file from the cursor, \keys{?} to do the opposite. \keys{n} to go to next result and \keys{N} to go to previous result.
    \end{itemize}{}
\end{frame}{}

\section{Writing}
\subsection{}

\begin{frame}{Insert Mode}
    \begin{enumerate}
        \item Make sure you are in normal mode. \keys{Esc}
        \begin{enumerate}
            \item \keys{i} to insert before cursor
            \item \keys{I} to insert at the start of line
            \item \keys{a} to insert after cursor
            \item \keys{A} to insert at the end of line
            \item \keys{o} to start a next line and insert
            \item \keys{O} to start a line above the current selection and insert
        \end{enumerate}{}
        \item Get out of insert once done. \keys{Esc}
    \end{enumerate}{}
\end{frame}{}

\begin{frame}{Making small edits - delete}
    In normal word, you can quickly delete a portion of text
    \begin{itemize}
        \item \keys{d + modifier}, deletes a certain portion based on the modifier
        \item \keys{dw} - delete word
        \item \keys{6dw} - delete 6 words
        \item \keys{dd} - delete the entire line
        \item \keys{d\$} - delete till end of line
        \item \keys{dt+char} - delete till character
    \end{itemize}{}
\end{frame}{}

\begin{frame}{Making small edits - change}
    Similarly, change allows you to quickly delete and change a certain portion of text
    \begin{itemize}
        \item \keys{c + modifier} - deletes then puts you into insert mode
        \item \keys{cw} - change word
        \item \keys{7cw} - change 7 words
        \item \keys{c\$} - change till end of line
        \item \keys{ct + char} - change till certain character
    \end{itemize}{}
\end{frame}{}

\begin{frame}{Making small edits - misc. (1/2)}
    \begin{itemize}
        \item \keys{y + modifier}, yank a certain portion and puts it in a put buffer (think of ctrl c)
        \item \keys{yy} - yank entire line
        \item \keys{yw} - yank word
        \item \keys{6yw} - yank 6 words
        \item \keys{yt + char} - yank till character
        \item \keys{p} - put/paste whatever was in the buffer
        \item \keys{P} - put/paste in the line above
    \end{itemize}{}
\end{frame}{}

\begin{frame}{Making small edits - misc. (2/2)}
    \begin{itemize}
        \item \keys{x} to delete a certain character
        \item \keys{r} to replace a character
        \item \keys{.} - do last action
        \item \keys{u} - undo action
        \item \keys{Ctrl + r} - redo action
    \end{itemize}{}
\end{frame}{}

\begin{frame}{Practice 1}
    \url{https://hckr.cc/ht-vim-p1}

    Try and fix the typos, and small errors here!
\end{frame}{}

\section{Intermediate Vim}
\subsection{}
\begin{frame}{Visual Mode}
    There are a few kinds of visual modes:
    \begin{itemize}
        \item Visual \keys{v}
        \item Visual Line \keys{V}
        \item Visual Block \keys{Ctrl + v}
    \end{itemize}{}{}
    We can use these selections along with the above commands: \keys{y}ank, \keys{d}elete, and \keys{c}hange.
\end{frame}{}

\begin{frame}{Opening Files and other commands}
    Aside from saving and quitting, there are a couple of pretty important commands to know:
    \begin{itemize}
        \item \keys{:enew} - opens a new file
        \item \keys{:e + filepath} - open the file at path
        \item \keys{sp} - open a new split
        \item \keys{vsp} - open a new vertical split
    \end{itemize}{}
\end{frame}{}

\begin{frame}{Macros}
    Macros are one of the really powerful features in vim that can significantly speed up your workflow.
    \begin{itemize}
        \item \keys{q + keystroke} to start recording a macro, \keys{q} again to stop recording
        \item \keys{@keystroke} to apply the macro
    \end{itemize}{}

\end{frame}{}
    
\begin{frame}{Practice 2}
    \url{https://hckr.cc/ht-vim-p2}

    It seems like the data from the first 3 columns are corrupted, let's remove them!

    \url{https://hckr.cc/ht-vim-p3}

    Let's use a macro to extract the names from these emails!
\end{frame}{}

\section{Conclusion}
\subsection{}

\begin{frame}{Conclusion}
    We covered the base fundamentals of operating vim. Here's some things we did not cover, but are really
    useful:
    \begin{itemize}
        \item \keys{di(} - delete everything inside the nearest brackets
        \item find and replace
        \item ...a lot more!
    \end{itemize}{}
\end{frame}{}

\begin{frame}{Extending Vim}
    One of the greatest features is that it is extremely customize and very easy to customize too!
    There are many great configs and plugins created for different workflows, which we can't cover in this
    session. Here are a few reccomendations:
    \begin{itemize}
        \item \url{https://github.com/amix/vimrc}
        \item \url{https://vimconfig.com}
        \item \url{https://github.com/gunbux/dotfiles}
    \end{itemize}{}
    There are also many forks of vim, notably neovim with much more features.
\end{frame}{}

\begin{frame}{Using vim outside of vim}
    Vim's hotkeys are very well loved, and many people love putting vimbindings on everything (myself included).
    Here are some you can look at:
    \item \href{https://chrome.google.com/webstore/detail/vimium/dbepggeogbaibhgnhhndojpepiihcmeb?hl=en}{Vimium - Vim for Chromium based browsers}
    \item \href{https://plugins.jetbrains.com/plugin/164-ideavim}{IdeaVim - Vim emulation for JetBrains}
\end{frame}{}

\begin{frame}{I need help!}
    Vim is a really powerful editor if you are able to master it. Don't worry, there are plenty of resources!
    \begin{itemize}
        \item \keys{:help command} - get manual for a specific command
        \item \keys{vimtutor} - built in vimtutor, give it a read, it shouldn't take more than 30 mins
        \item \href{https://www.vimgolf.com}{VimGolf} - really good practice: edit the file in the minimum strokes
        \item There are tons of vim wikis, guides and cheatsheets out there, just search for them!
    \end{itemize}{}
\end{frame}{}

\begin{frame}
  \frametitle{Talk to us!}
  \begin{itemize}
    \item \textbf{Feedback form}: \url{https://hckr.cc/ht-fb10}
    \item \textbf{Upcoming Hacker Tools}:
          Tue, 25th Oct 2022, 6.30pm -- WSL: \url{https://hckr.cc/ht-11}
          
    \item \textbf{Upcoming Hackerschool}:
          Sat, 22nd Oct 2022, 10am -- Flask: \url{https://hckr.cc/hs2223s1-week10-signup}
  \end{itemize}
\end{frame}

\end{document}